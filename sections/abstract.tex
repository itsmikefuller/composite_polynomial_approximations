\begin{abstract}
\thispagestyle{plain}
\pagenumbering{roman}

Approximation theory is ubiquitous in scientific computing, in particular in the computation of matrix functions. For instance, the matrix sign function $\sgn(A)$ and matrix roots $A^{1/p}$ can be approximated by matrix iterations such as Newton's Method. Efforts have been made to improve the efficiency of Newton's Method in modern mathematical research by considering scaled iterative processes.

\bigskip{}

Matrix iterations can effectively be viewed as approximating a matrix function $f(A)$ by a composite rational function of $A$. Polynomials are often overlooked for their comparatively weaker convergence properties, but at a matrix argument they are much cheaper to compute than rational functions, since matrix polynomials are inversion-free.

\bigskip{}

Composing low-degree polynomials or rational functions is an effective way of rapidly producing such functions of much higher degree, and a remarkable result in rational approximation theory states that appropriately composing Zolotarev functions generates higher-order Zolotarev functions. This dissertation investigates the polynomial analogue by constructing a composite polynomial approximation to the sign function based on a greedy algorithm. We show that our construction is equivalent to a scaled Newton-Schulz iteration, and analyse its convergence with respect to the degrees of freedom of the approximation. Moreover, we show that the iteration can be used to obtain a novel composite polynomial approximation to the square root function.

\end{abstract}